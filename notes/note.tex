%%%%%%% P R E A M B U Ł A %%%%%%%
% W razie potrzeby marginesy dokumentu możesz zmienić w linii 31, odblokowując ją poprzez usunięcie znaku % na jej początku.
% Właściwa edycja treści rozpoczyna się w linii 48.

\documentclass[12pt,a4paper]{article}

\usepackage[MeX]{polski}
\usepackage[utf8]{inputenc}
\usepackage{fontenc}
\usepackage[english,polish]{babel}
\usepackage{cite}
\usepackage{amsmath,amsfonts}
\usepackage{graphicx}
\usepackage[tight,footnotesize]{subfigure}
\usepackage{listings}
\usepackage{xcolor}
\usepackage[small]{caption}
\usepackage{makecell} % Used to break text in a single cell of a table.
\usepackage{float}
\restylefloat{table}
\usepackage{tabto}
\usepackage[shortlabels]{enumitem}
\usepackage{adjustbox}
\usepackage{array}
\usepackage{fancyvrb}
\usepackage[colorlinks=true,citecolor=blue,linkcolor=red,urlcolor=blue,pagebackref=true]{hyperref}
\usepackage[all]{nowidow}
\usepackage{upquote} % Rozwiązuje problem zakręconych cudzysłowów w otoczeniu verbatim, gdzie wystarczy użyć '. Natomiast w {\tt } trzeba zamiast ' pisać \textquotesingle, czyli np. {\tt \textquotesingle Ala ma kota.\textquotesingle}. Inaczej będą zakręcone, nie proste, co bruździ przy przeklejaniu kodu z pdf do edytora kodu.
\frenchspacing

%\usepackage[top=35mm,left=35mm,bottom=35mm,right=35mm]{geometry}

% Potrzebne do tworzenia schematów blokowych.
% https://www.overleaf.com/learn/latex/LaTeX_Graphics_using_TikZ:_A_Tutorial_for_Beginners_(Part_3)%E2%80%94Creating_Flowcharts
\usepackage{tikz} 
\usetikzlibrary{shapes.geometric, arrows} 
\tikzstyle{startstop} = [ellipse, minimum width=3cm, minimum height=1cm, text centered, draw=black, fill=black!0]
\tikzstyle{io} = [trapezium, trapezium left angle=70, trapezium right angle=110, minimum width=3cm, minimum height=1cm, text centered, draw=black, fill=yellow!0]
\tikzstyle{process} = [rectangle, minimum width=3cm, minimum height=1cm, text centered, draw=black, fill=green!0]
\tikzstyle{decision} = [diamond, minimum width=3cm, minimum height=1cm, text centered, draw=black, fill=orange!0]
\tikzstyle{arrow} = [thick,->,>=stealth]


%%%%%%%%% D O K U M E N T %%%%%%%%%

\begin{document}

\title{Blog Project}

\author{Zofia Dobrowolska}

\date{\today}

\maketitle

\section{Create virtual environment}
Project created for deployment needs a dedicated virtual environment to run in production. Step-by-step instruction on how to do it for Django project.\\
\begin{enumerate}
\item Navigate to root of a project:
	\begin{verbatim}
	~ cd <<project path>>
	\end{verbatim}
\item Create the virtual environment .venv:
	\begin{verbatim}
	~ python -m venv <<name of the virtual environment>>
	\end{verbatim}
\item Activate the venv:
	\begin{verbatim}
	~ venv\Scripts\activate
	\end{verbatim}
\item Install django:
	\begin{verbatim}
	~ pip install django
	\end{verbatim}
\item Create the Django project:
	\begin{verbatim}
	~ django-admin startproject <<name>>
	\end{verbatim}
\item Check by running the server (this command starts the project, making it accessible from web):
	\begin{verbatim}
	~ python manage.py runserver	
	\end{verbatim}
\item Add an app to the project:
	\begin{verbatim}
	~ django-admin startapp <<name>>	
	\end{verbatim}
\item Migrate changes:
	\begin{verbatim}
	~ python manage.py makemigrations
	~ python manage.py migrate
	\end{verbatim}
\item Create an admin (user):
\item Migrate changes:
	\begin{verbatim}
	~ python manage.py createsuperuser
	\end{verbatim}
Then you will be asked to enter the user name, password, and email address. You can omit the email, though.
\item
\end{enumerate}


\section{Handleing venv}
Storing a Django project and its virtual environment objectively consumes too much space. Yet, for deployment, you have to know the requirements for the project to work. You can generate \textit{requirements.txt} by typing in git BASH a command\footnote{\href{https://pip.pypa.io/en/stable/cli/pip_freeze/}{pip freeze}}:
\begin{verbatim}
env1/bin/python -m pip freeze \> requirements.txt
\end{verbatim}
and to install from it to another environment run command:
\begin{verbatim}
env2/bin/python -m pip install -r requirements.txt
\end{verbatim}

Then include \textit{.venv} in your \textit{.gitignore} file.


\section{.gitignore}
The \textit{.gitignore} file is a text file that tells Git which files or folders to ignore in a project\footnote{\href{https://www.freecodecamp.org/news/gitignore-what-is-it-and-how-to-add-to-repo/}{.gitignore}}. A local \textit{.gitignore} file is usually placed in the root directory of a project, but you can also create a global one. Mind that, entries in the global file will be ignored in all your Git repositories.\\
To create a local \textit{.gitignore} file, create a text file and name it \textit{.gitignore} (remember to include the . at the beginning). Then edit this file as needed. Each new line should list an additional file or folder that you want Git to ignore.\\

The entries in this file can also follow a matching pattern.
\begin{itemize}
	\item * is used as a wildcard match
	\item / is used to ignore pathnames relative to the .gitignore file
	\item \# is used to add comments to a .gitignore file 
\end{itemize}
Par exemple:
\begin{verbatim}
# Ignore Mac system files
.DS_store

# Ignore node_modules folder
node_modules

# Ignore all text files
*.txt

# Ignore files related to API keys
.env

# Ignore SASS config files
.sass-cache
\end{verbatim}

To add or change your global .gitignore file, run the following command:

\begin{verbatim}
To add or change your global .gitignore file, run the following command:
\end{verbatim}

This will create the file  \textit{\~/gitignore\_global}. Now you can edit that file the same way as a local \textit{.gitignore} file. All of your Git repositories will ignore the files and folders listed in the global \textit{.gitignore} file.

To untrack a single file, ie stop tracking the file but not delete it from the system use:
\begin{verbatim}
git rm --cached filename
\end{verbatim}
To untrack every file in .gitignore:

First commit any outstanding code changes, and then run:
\begin{verbatim}
git rm -r --cached
\end{verbatim}
This removes any changed files from the index(staging area), then run:
\begin{verbatim}
git add .
\end{verbatim}
Commit it:
\begin{verbatim}
git commit -m ".gitignore is now working"
\end{verbatim}
To undo \begin{verbatim}git rm --cached filename\end{verbatim}, use git add filename








































\end{document}